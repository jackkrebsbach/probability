 \documentclass[10pt]{report}
\input{preamble}
\input{macros}
\input{letterfonts}

\title{ Notes for Introduction to Probability }

\author{Jack Krebsbach}
\date{\today}


\begin{document}
\maketitle

\setcounter{secnumdepth}{1}
\chapter{Set Theory}

% \section{Set Theory}
\section{Definitions}

\dfn{Set}{
  A set is a collection of objects. The objects in a  set are frequently called \textbf{elements.}
}


\section{Set Properties}

We can think of 'distributing' the set operations in the parentheses.

\begin{itemize}
  \item $A \cap (B \cup C) = (A \cap B) \cup (A \cap C) $
  \item $A \cup (B \cap C) = (A \cup B) \cap (A \cup C) $
  \item In general ($\bigcup_{i=1}^{\infty} A_i)^C = \bigcap_{i=1}^{\infty} A_i^C$
  \item $n(A \cup B) = n(A) + n(B) - n(A \cap B)$
  \item If $A \subseteq B$ then $n(A) \leq n(B)$
  \item Every set contains the empty set $\emptyset$


    
\end{itemize}


\ex{16}{
  Let $A$ and $B$ be any two sets. Use Venn diagrams to show that $B = (A \cap B) \cup (A^C \cap B^C)$ and $A \cup B= A \cup (A^C \cap B)$

These are disjoint or mutually exclusive unions.
}

\ex{}{
A survey of a group's viewing habits over tne last yun following information.
    
(i) $28 \%$ watched gymnastics (G)
(ii) $29 \%$ watched baseball (B)
(iii) $19 \%$ watched soccer (S)
(iv) $14 \%$ watched gymnastics and baseball
(v) $12 \%$ watched baseball and soccer
$10 \%$ watched gymnastics and soccer
 $8 \%$ watched all three sports.


Represent the statement "the group that watched none of the three sports during the last year" using operations on sets.
  A survey of a groups viewing habits over the last year revealed the following information.

  $$(G \cup B \cup S)^C$$
}


\ex{18}{
  Mutually exclusion principle.
  Show that if $A,B,C$ are subsets of the universe $U$ then $n(A\cap B\cap C) = n(A) + n(B) + n(C) - n(A\cap B) - n(B\cap C) - n(C\cap A) - n(A \cap B \cap C)$

  $\begin{aligned} & n(A)+n(B)+n(C)-n(A|| B)-m(1+\cdots \\ & \text { - }(A \cup B \cup C)=n(A \cup(B \cup C))=n(A)+n(B \cup C)-n(A \cap(B \cup C)) \\ & =n(A)+n(B)+n(C)-n(B \cap C)-n[(A \cap B) \cup(A \cap C)] \\ & =n(A)+n(B)+n(C)-n(B \cap C)-\left[\begin{array}{l}n(A \cap B)+n(A \cap C) \\ -n(A \cap B \cap A \cap C)\end{array}\right] \\ & =n(A)+n(B)+n(C)-n(A \cap B)-n(A \cap C)-n(B \cap C) \\ & +n(A \cap B \cap C) \# \\ & \end{aligned}$
}

\subsection{DeMorgan Laws}

\dfn{DeMorgan Laws}{

\begin{enumerate}
  \item  $(A \cup B)^C = A^C \cap B^C$
  \item $(A \cap B)^C = A^C \cup B^C$

\end{enumerate}
}

    
\begin{enumerate}
    
  \item  $(A \cup B)^C = A^C \cap B^C$
    \begin{myproof}
        
    \par
    $\Rightarrow$ Let $x \in (A \cup B)^C$. Then $x \not\in (A \cup B)$ and it follows $x \not\in A$ and $x\not\in B$.This means that $x \in A^C$ and $x \in B^C$. Thus, $x \in A^C \cap B^C.$
    \par
    $\Leftarrow$ Let $x\in A^C \cap B^C$. Then $x \in A^C$ and $x \in B^C.$ This means $x \not\in A $ and $x \not\in B$. Finally, $x\not\in A \cup B$ which implies $x \in (A \cup B)^C.$

    \end{myproof}

  \item $(A \cap B)^C = A^C \cup B^C$

    \begin{myproof}
      $\Rightarrow$ Let $x \in (A \cap B)^C$. Then $x \not\in A \cap B$ and it follows that $x \not\in A$ or $x \not\in B.$Then $x \in A^C$ or $x \in B^C.$ Thus, $x \in A^C \cup B^C$.
      \par
      $\Leftarrow$ Let $x \in A^C \cup B^C.$ Then $x \in A^C$ or $x \in B^C$ and it follows $x \not\in A$ or $x\not\in B.$ Then $x$ can not be in their intersection so $x \not\in A \cap B.$ Finally, this means $x \in (A \cap B)^C.$
        
    \end{myproof}

\end{enumerate}


\section{Counting}

\subsection{Multiplication Principle}

If a compound action can be broken into a series of $k$ compoonent actions each of these can be performed in $n_1, n_2, \dots, n_k $  ways respectively, then the compound action can be performed $n_1n_2n_3 \dots n_k$ ways.

\smallskip

\ex{}{
How many license plates with 3 letters followed by 3 digits exist?

\sol    A 6-step process. (1) Choose the first letter (2) choose teh second letter

    $26^3 10^3= 125,760,000$
}

\ex{}{
  How many numbers in the range 1000-9999 have no repeated digits?
  \bigskip

  \sol Usually solve the complement problem but in this case the original form is easier.

  The solution is $$9(9)(8)(7)(6) = 4,536$$
}

\ex{Class problem}{
How many different ways can you play an album of 15 songs.


\sol $$ 15!$$
}


\subsection{Permutation Principle}

It is useful to define $0! = 1$ because it is more convenient to do so.

Suppose I have $n$ \textbf{distinct} objects and I wish to select $r$of them, and the order in which I select them is important. 


\dfn{Permutation}{

  $${}_{n} P_r = n(n-1)(n-2)\dots(n-(r-1)) =  \frac{n!}{(n-r)!}$$

}

 Specifically the number of ways to arrange $n$ \textbf{distinct} objects where the \textbf{order} in which they are selected. This means that $n=r.$

 \ex{23}{

How many license plates are there that start with three letters followed by 4 digits (no repetition).

\sol
Choices of $3$ letters and choices of $3$ numbers 
$$ 26(25)(24) * 10(9)(8)  = 786,240,000 \text{ number of choices }$$
 }

 \ex{}{
   How many five digit zip codes can be made where all digits are different? The possible digits $0$ through $9.$

\sol ${}_10 P_5$
 }


\subsection{Combination Principle}

Suppose I have $n$ \textbf{distinct} objects and I wish to select $r$of them, and the order in which I select them is \textit{not} important. We just divide the number of combinations we can take $r$ where order is important.

\dfn{}{
  $${}_n C_r = \frac{{}_n P_r }{r!} = \frac{n!}{r!(n-r)!}$$
}


\dfn{Combination}{

}


\subsection{Binomial Theorem}

If $x$ and $y$ are variables and $n$ is a non-negative integer, then $$(x +y)^n = \sum_{k=0}^{n} C_k^n x^{n-k} y^k$$

\ex{}{
  A jury consisting of $2$ women and $3$ men is to be selected form a group of $5$ women and $7$ men. In how many different ways can this be done?
  Suppose that either Steve or Harry must be selected but not both, then in how many ways this jury can be formed?

\sol
  \begin{enumerate}
    \item  $$ {}_{5}C_2 \text{ } {}_7{C}_3 = 350 \text{ ways }$$
    \item If Steve is in, then there are $5$ mean left to choose from. ${}_5 C_2 = 10(10)=100.$ Same argument for when Harry must be in, there are 100 ways. So total number of number of ways is $100 +100 =200$ ways.
  \end{enumerate}

}

\ex{}{

  \begin{itemize}
    \item  How many ways can $6$ people line up for a picture.
      \sol ${}_6 P_6 = 6!$
    \item Can they choose a president and a secretary? 
      \sol ${}_6 P_2.$
    \item Can they choose three member to attend a state conference with no regard to order
      \sol ${}_6 C_3.$
      
  \end{itemize}
}

\ex{}{
  \begin{itemize}
    \item Choose 3 out of 10 songs = ${}_{10} C_{3}$
    \item Rate top 3 songs $= {}_10 P_{3}$
  \end{itemize}
}

\ex{}{
  We have six distinct CDs. How many ways would there be to place 3 in one box and $3$ in another. (be careful, do we know the boxes are distinct?)

  \sol $ {}_6 C_3 ({}_3 C_3)$
}

\ex{}{
Out of 10 distinct parts, 3 are defective. We select 2 of these parts at random. How many ways are there so that 

\begin{itemize}
  \item none of the 2 is defective. ${}_7 C_2 ({}_3C_0)$
  \item Exactly 1 is defective and 1 is not defective ${}_3 C_1 ({}_7C_1)$
  \item Both are defective ${}_3 C_2$

\end{itemize}
}

\dfn{Sample Space}{
  The sample space S of an experiment is the set of all possible outcomes for the experiment
}

\dfn{Event}{
  Subset of the sample space.
}

\dfn{Probability}{
  Is the measure of occurrence of an event.  We denote the probability of event A happening $PR(A).$ We can think of it as a function.
}

The function Pr satisfies the following axioms, known as Kolmogorov axioms:

\begin{itemize}
  \item (a) Axiom 1: For any event $E, 0 \leq \operatorname{Pr}(E) \leq 1$.
  \item (b) Axiom 2: $\operatorname{Pr}(S)=1$.
  \item (c) Axiom 3: For any sequence of mutually exclusive events $\left\{E_n\right\} n \geq 1$, that is $E_i \cap E_j=\phi$ for $i \neq j$, we have $\operatorname{Pr}\left(\cup_{i=1}^{\infty} E_i\right)=\Sigma_{i=1}^{\infty} \operatorname{Pr}\left(E_i\right)$. This axiom is known as countable additivity.
i. Specifically, for mutually exclusive events $A, B$ and $C$, we have
    
\end{itemize}

\end{document}
